%%%%%%% Preambule %%%%%%%%
% třída dokumentu
\documentclass[12pt,a4paper,hidelinks]{article}

% kvůli písmům a ligaturám (např. --)
% toto je pro XeLaTeX!
\usepackage{fontspec}
\defaultfontfeatures{Mapping=tex-text}
\usepackage{xunicode}
\usepackage{xltxtra}
\setmainfont{Times New Roman}

% nastavení jazyka (kvůli dělení slov, apod.)
% toto je pro XeLaTeX!
\usepackage{polyglossia}
\setdefaultlanguage{czech}

% když polyglossia, fontspec, xltxtra je to xelatex

% rozšíření od AMS na vzorečky
\usepackage{amsmath}
\usepackage{amsfonts}
\usepackage{amssymb}

% definice nové fce
\DeclareMathOperator{\tg}{tg}

%balíček pro \dif
%\usepackage{commath}
%pro rovná řec. písmena
%\usepackage{upgreek}

% potřebujeme na obrázky
\usepackage{graphicx}

% okraje
% \usepackage[left=2cm,right=2cm,top=2cm,bottom=2cm]{geometry}
\usepackage{a4wide}

% alternativní písma
% \usepackage{kpfonts-otf}
% \usepackage{fourier-otf}
 \setmainfont{Times New Roman}


% balíček pro výplňový text (lorem ipsum)
\usepackage{lipsum}

% balíček pro lepší práci s odkazy
\usepackage[unicode]{hyperref}

% balíček pro nastavení odstavců
\usepackage{parskip}
\setlength{\parindent}{2em}
% balíček pro nastavení řádkování
\usepackage{setspace}
\onehalfspacing

% balíček pro rozšíření seznamů
\usepackage{enumitem}
\setlist[enumerate, 2]{label={\alph*)}}
\setlist[enumerate, 3]{label={\roman*)}}

%bal. pro sluč řádků
\usepackage{multirow}

%bal. pro hezčí tabulky
\usepackage[table]{xcolor}
\usepackage{booktabs}

%pro hezčí popisky
\usepackage{caption}
\captionsetup [table]{skip=2pt}

%bal pro sazbu do vícce sloupců
\usepackage{multicol}

%české uvozovky
\usepackage{csquotes}

\usepackage{footnote}
\makesavenoteenv{figure}

% odřezávání rádků
\tolerance=1
\emergencystretch=\maxdimen
\hyphenpenalty=10000
\hbadness=10000

%balíček pro hezčí práci s čísly a jednotkami
%pro příkazy \num{1235456}, \SI{123456}{km} či \SI{30}{\celsius}
\usepackage{siunitx}
\sisetup{locale=DE}

%Numbering věc i guess
\usepackage{chngcntr}
\counterwithin*{section}{part}


%balíček pro automatické generování citací
% urldate: formátovaní data citování (položka urladate)
% iso je yyyy-mm-dd
% sorting: způsob řazení citací, none je původní (citované) pořadí
% nty: jméno, název, rok
% style: způsob formátování/zobrazení citace, iso-numeric je čsn iso 690-2 (citace.com + číslované), iso-author title je to stejný, ale pro humanitní (pozn. pod čarou)
% pro přírodní vědy
% \usepackage[urldate=iso, sorting=none, style=iso-numeric]{biblatex}
% pro humanity
 \usepackage[urldate=iso, sorting=nty, style=iso-authortitle]{biblatex}
\addbibresource{ZdrojeAndShit.bib}

% Autor, název a datum
\author{Vojtěch Hána}
\title{Jednoduchá PC hra}
\date{mnogo}

%%%%%%% Konec preambule = začátek dokumentu %%%%%%%%

\begin{document}
\begin{titlepage}
	\centering
	{\large Gymnázium Nad Štolou 1, Praha 7}
	\vfill
	{\Large \textsc{Maturitní Práce}} \par
	\scalebox{1.4}{{\Huge Jednoduchá počítačová hra}}
	\vfill\vspace{2cm}
	
	%\begin{large}
	\begin{tabular}{rl}
	Autor práce: & Vojtěch Hána \\
	Vedoucí práce: & Martin Sourada\\  
	Třída: & 5S \\ 
	\addlinespace
	\multicolumn{2}{c}{2021/2022} \\ 
	\end{tabular} 
	%\end{large}
\end{titlepage}
\addtocounter{page}{1}

\clearpage
\thispagestyle{empty}

Prohlášení: Prohlašuji, že jsem maturitní práci vypracoval samostatně, použil jsem pouze podklady uvedené v přiloženém seznamu a postup při zpracování a dalším nakládání s prací je v souladu se zákonem č. 121/2000 Sb., o právu autorském, o právech souvisejících s právem autorským a o změně některých zákonů (autorský zákon), ve znění pozdějších předpisů.\\
\\
V Praze dne \today\\

\clearpage
\thispagestyle{empty}
Poděkováníííííííííííííííííí\\
\clearpage
%\begin{small}
	\tableofcontents
%\end{small}
\clearpage

\section*{Úvod}
\addcontentsline{toc} {section} {Úvod}
Já ti jednu uvedu, jen počkej!
\clearpage

\part{Teoretická část}

\section{Charakteristika hry}
Ano taťko unreale break my node links.
\clearpage

\section{Užité nástroje}
\subsection{Engine}
Veřejnosti je přístupno nepřeberné množství herních enginů, každý z nich se svými plusy a mínusy, orientovaný na jiný segment trhu. Například \textit{Unigine} exceluje ve velkých scénách díky 64bitovým souřadnicím, zatímco \textit{Clausewitz} je zaměřený na top-down grand-strategy hry.\\
Jelikož Astra je mým prvním velkým projektem, co se vývinu her týče, bylo pro mne důležité, aby pro zvolený engine byla dostupná rozsáhlá a kvalitní dokumentace a aktivní komunita, na kterou by se bylo možná obrátit v případě potíží. Tím z výběru vypadávají všechny málo známé enginy a také ty, které sice známé jsou, ale většinou na nich vyvíjejí pouze velká studia, jako například \textit{CryEngine}.\\
Po tomto prvním kroku zbývali dva seriózní kandidáti, a to \textit{Unity} a \textit{Unreal Engine}. Finální volba nakonec padla na novější Unreal Engine 5 (nadále v této práci zkracován UE). Ten se jevil jako lepší volba díky jeho systémům jako je například engine osvětlení Lumen, který by mi značně usnadnil tvorbu realisticky vypadající grafiky.\\

\subsection{Další užité nástroje}
UE má ve své základní distribuci zabudováno rozsáhlé množství nástrojů pro tvorbu a nakládání s různými soubory. I přesto bylo v průběhu vývoje nutné využít několik dalších programů:\\
\subsubsection{GIMP}
GNU Image Manipulation Program (GIMP) je bezplatný rastrový grafický editor, který slouží k úpravě a tvorbě obrázků. Tento program jsem využil na tvorbu některých textur a design prvků uživatelského rozhraní.\\
Oproti konkurenčním programům, např. Adobe Photoshop, jsem GIMP zvolil zejména kvůli předchozí znalosti tohoto programu.\\
\subsubsection{Audacity}
Audacity je open-source program pro úpravu zvukových souborů. Vzhledem k možnostem UE základně upravovat zvukové stopy (hlasitost, výška, filtry) přímo v enginu jsem Audacity užil jen zřídka a to na úpravu hlasových stop pro postavu hráčovy AI pomocnice a pro přípravu dalších zvukových stop převodem na podporovaný kodek.\\
\subsubsection{Git}
Git je distribuovaný systém správy verzí, který se používá pro sledování změn v kódu a koordinaci práce více lidí na společném projektu. Je to nástroj, který umožňuje uživatelům ukládat, verzovat a spravovat svůj kód a jeho historii.\\
Git umožňuje uživatelům pracovat na kopii kódu, kterou si mohou upravovat bez toho, aby se to projevilo na hlavní větvi (tzv. "master branch"). Poté, co provedou své změny a jsou s nimi spokojeni, mohou je zahrnout do hlavní větve. Git také umožňuje ukládat různé verze kódu v různých větvích, což usnadňuje vývoj a testování nových funkcí nebo náprav chyb. Díky němu je také snadnější zpětné získání předchozích verzí kódu a práce na nich.\\
Jelikož jsem na projektu pracoval sám, řady z vlastností Gitu jsem vůbec nevyužil. Stejně tak zůstala nevyužitá i možnost vývoje ve více větvích (z důvodu, že jsem dříve nebyl se systémy sledování verzí), což není optimální.\\
\subsubsection{Blender}
Blender je otevřený a zdarma dostupný software pro 3D modelování, animaci a vizualizaci, široce používaný v oblasti filmového a herního průmyslu, architektury, designu a mnoha dalších. Uživatelům umožňuje vytvářet 3D objekty, animovat je, texturovat, osvětlovat a renderovat. Další funkcí Blenderu jsou například simulace fyziky, jako jsou srážky a deformace, simulace tekutin a plamenů apod.\\
Uživatelé také mohou vytvářet složité animace pomocí animačních klíčů, sledování kamer a světel, a dalších funkcí. Právě tuto konkrétní funkci jsem při vývoji použil, a to na vytvoření krátkého videa s logem hry, které se přehrává, když je hra spuštěna.\\


\clearpage

\section{Vývoj}

\clearpage

\part{Uživatelská příručka}

\section{Cíl hry}

\clearpage

\section{Ovládání}

\clearpage


\clearpage
\section*{Závěr}
\addcontentsline{toc} {section} {Závěr}
Závěr? Jakýpak závěr? Nejde ti snad otevřít zavařovací sklenice, drahý TeXu?

\clearpage

\section*{Resumé}
\addcontentsline{toc} {section} {Resumé}
Tato práce dělá brrrrrrr\\
\\
This paper goes brrrrrrr\\


\clearpage

\section*{Příloha}
\addcontentsline{toc} {section} {Příloha}
\textit{Pozn.: Pokud není uveden zdroj, média jsou ve veřejné doméně.}
\clearpage


\printbibliography[title={Seznam literatury a zdrojů}]
\addcontentsline{toc} {section} {Seznam užité literatury a zdrojů}

\end{document}
